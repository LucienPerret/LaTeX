\documentclass[10pt,a4paper,landscape]{article}
\usepackage[utf8]{inputenc}
\usepackage[ngerman]{babel}
\usepackage[T1]{fontenc}
\usepackage{amsmath}
\usepackage{amsfonts}
\usepackage{amssymb}
\usepackage{graphicx}
\usepackage{lmodern}
\usepackage{physics}
\usepackage[left=0.5cm,right=0.5cm,top=0.5cm,bottom=0.5cm]{geometry}
\usepackage{siunitx}
\usepackage{fancyhdr}
\usepackage{enumerate}
\usepackage{mhchem}
\usepackage{mathtools}
\usepackage{graphicx}
\usepackage{float}
\usepackage{xcolor}
\usepackage[framemethod=tikz]{mdframed}
\usepackage{csquotes}
\usepackage{trfsigns}
\usepackage{capt-of}
\usepackage{comment}
\usepackage{tabularray}
\usepackage{multicol}
\usepackage{titlesec}
\titlespacing{\section}{0pc}{1ex}{1ex}
\columnseprule=0.4pt

\definecolor{MyCol1}{HTML}{E9A27C}
\definecolor{MyCol2}{HTML}{E7BBA6}
\definecolor{MyCol3}{HTML}{8CB6BC}
\definecolor{MyCol4}{HTML}{0E8992}
\definecolor{MyCol5}{HTML}{0D313A}


\newmdenv[backgroundcolor=black!10,align=center,roundcorner=8pt,hidealllines=true,nobreak,
	leftmargin=0cm,rightmargin=0cm,innertopmargin=0.15cm,innerbottommargin=0.15cm,skipabove = 0,
	skipbelow = 0]
	{standard}

\newenvironment{Standard}{\standard\centering}{\endstandard}


\begin{document}{}
\begin{multicols*}{3}

\section*{\centering Standard Derivative \& Integral}
\begin{Standard}
	\resizebox*{\textwidth}{!}{
	\def\arraystretch{1.5} 
	\begin{tabular}{|c|c|c|}
		\hline
		Function | f(x) & Derivative | f'(x) & Integral | F(x)\\
		\hline
		\(1\) & \(0\) & \(x+C\) \\
		\hline
		\(x\) & \(1\) & \(\frac{1}{2}x^2+C\) \\
		\hline
		\(\frac{1}{x}\) & \(-\frac{1}{x^2}\) & \(\ln|x|+C\) \\
		\hline
		\(x^a \: with \: a \: \in \: \mathbb{R}\) & \(ax^{a-1}\) & \(\frac{x^{a+1}}{a+1}+C\) \\
		\hline
		\(\sin(x)\) & \(\cos(x)\) & \(-\cos(x)+C\) \\
		\hline
		\(\cos(x)\) & \(-\sin(x)\) & \(\sin(x)+C\) \\
		\hline
		\(\tan(x)\) & \(1+\tan^2{(x)=\frac{1}{\cos^2{(x)}}}\) & \(-\ln|\cos(x)|+C\) \\
		\hline
		\(\cot(x)\) & \(-1-\cot^2{(x)}=-\frac{1}{\sin^2{(x)}}\) & \(\ln(\sin(x))+C\) \\
		\hline
		\(e^x\) & \(e^x\) & \(e^x+C\) \\
		\hline
		\(a^x\) & \(\ln(a)\cdot a^x\) & \(\frac{a^x}{\ln(a)}+C\) \\
		\hline
		\(\ln(x)\) & \(\frac{1}{x}\) & \(x\ln(x)-x+C\) \\
		\hline
		\(\log_a(x)\) & \(\frac{1}{x\ln(a)}\) & \(x\log_a(x)-\frac{x}{\ln(a)}+C\) \\
		\hline
		\(\arcsin(x)\) & \(\frac{1}{\sqrt{1-x^2}}\) & \(x\arcsin(x)+\sqrt{1-x^2}+C\) \\
		\hline
		\(\arccos(x)\) & \(-\frac{1}{\sqrt{1-x^2}}\) & \(x\arccos(x)-\sqrt{1-x^2}+C\) \\
		\hline
		\(\arctan(x)\) & \(\frac{1}{1+x^2}\) & \(x\arctan(x)-\frac{1}{2}\ln(1+x^2)+C\) \\
		\hline
	\end{tabular}
	}
\end{Standard}

\section*{\centering Partial Fraction Decomposition}
\begin{Standard}
	\begin{enumerate}
		\item The fractions numerator needs to be of a lower \\ polynomial degree than the denominator	
		\item Guess one of the Zeros with the values: \\ \((-3,-2,-1,1,2,3)\) \\
		which will t
		\item
	\end{enumerate}
\end{Standard}

\section*{\centering Horners Method}
\begin{Standard}
	begin by guessing one of the zero points with the values:
	\((-3,-2,-1,1,2,3)\) \\
	\vspace{10pt}
	\(f(x)=5x^3-8x^2-27x+18 \rightarrow \textcolor{MyCol4}{x=-2}\) \\
	\vspace{10pt}
	factors before x go into table \\
	\(\downarrow \) \\ 
	\resizebox*{\textwidth}{!}{
	\begin{tabular}{cccc}
		\(5\) & \(-8\) & \(-27\) & \(18\) \\
		\rule{0pt}{3ex}	
		\( \; \)	& \(5 \cdot \textcolor{MyCol4}{(-2)}=-10\) & \(-18\cdot \textcolor{MyCol4}{(-2)}=36\) &
		\(9\cdot \textcolor{MyCol4}{(-2)}=-18\) \\
		\rule{0pt}{3ex}	
		\(5\) & \(-8+(-10)=-18\) & \(-27+36=9\) & \(18+(-18)=0\) \\
	\end{tabular}
	}
\end{Standard}

\end{multicols*}
\end{document}
