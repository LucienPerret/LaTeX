\documentclass[10pt,a4paper]{article}
\usepackage[utf8]{inputenc}
\usepackage[ngerman]{babel}
\usepackage[T1]{fontenc}
\usepackage{amsmath}
\usepackage{amsfonts}
\usepackage{amssymb}
\usepackage{graphicx}
\usepackage{lmodern}
\usepackage{physics}
\usepackage[left=1cm,right=1cm,top=2cm,bottom=1.5cm]{geometry}
\usepackage{siunitx}
\usepackage{fancyhdr}
\usepackage{enumerate}
\usepackage{mhchem}
\usepackage{mathtools}
\usepackage{graphicx}
\usepackage{float}
\usepackage{xcolor}
\usepackage{mdframed}
\usepackage{csquotes}
\usepackage{trfsigns}
\usepackage{capt-of}

\mdfdefinestyle{exercise}{
	backgroundcolor=black!10,align=cente,roundcorner=8pt,hidealllines=true,nobreak,
}

\begin{document}
\twocolumn
 \section*{Standard Ab/Aufleitungen} 
\begin{mdframed}[style=exercise]
\begin{align*}
	\begin{tabular}{|c|c|c|}
		\hline
		function & derivative & integral\\
		\hline
		x & 1 & _\\
		\hline
		x^a | a \epsilon \numberset{R} & ax^{a-1} & _ \\
		\hline
		\sin(x) & \cos(x) & -\cos(x)\\
		\hline
		\cos(x) & -\sin(x) & \sin(x)\\
		\hline
		\tan(x) & \frac{1}{\cos^2(x)} & -\ln(\cos(x))\\
		\hline
		\cot(x) & -\frac{1}{\sin^2(x)} & \ln(\sin(x))\\
		\hline
	\end{tabular}
\end{align*}
\end{mdframed}
\end{document}