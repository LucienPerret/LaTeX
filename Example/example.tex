\documentclass[10pt,a4paper]{article}
\usepackage[utf8]{inputenc}
\usepackage[T1]{fontenc}
\usepackage{amsmath}
\usepackage{amsfonts}
\usepackage{amssymb}
\usepackage{graphicx}
\usepackage{lmodern}
\usepackage{physics}
\usepackage[left=1cm,right=1cm,top=2cm,bottom=1.5cm]{geometry}
\usepackage{siunitx}
\usepackage{fancyhdr}
\usepackage{enumerate}
\usepackage{mhchem}
\usepackage{mathtools}
\usepackage{graphicx}
\usepackage{float}
\usepackage{xcolor}
\usepackage{mdframed}
\usepackage{csquotes}
\usepackage{trfsigns}
\usepackage{capt-of}

\sisetup{locale=DE}
\sisetup{per-mode = symbol-or-fraction}
\sisetup{separate-uncertainty=true}
\DeclareSIUnit\year{a}
\DeclareSIUnit\clight{c}
\mdfdefinestyle{exercise}{
	backgroundcolor=black!10,roundcorner=8pt,hidealllines=true,nobreak
}

\begin{document}
\twocolumn
\pagestyle{fancy}
\lhead{Einführung in die digitale Signalverarbeitung \\ Formelsammlung}
\rhead{\today \\ Elsner, Jean}
\section{Lineare zeitinvariante Systeme}
  \subsection{Eigenschaften}
  Eigenschaften LTI-Systeme
  \begin{mdframed}[style=exercise]
    \begin{enumerate}
      \item Stabilität\\
      \(\abs{x(t)} < M < \infty \Rightarrow \abs{y(t)} < N < \infty\)
      \item Linearität\\
      \(W\qty{\sum_{k=1}^{N}a_n x_n(t)}=\sum_{n=1}^{N}W\qty{a_n x_n(t)}\)
      \item Zeitinvarianz\\
      \(W\qty{x(t-t_0)} =y(t-t_0)\)
      \item kausalität\\
      $t < 0 \Leftarrow x(t)=0 \land y(t)=0$
    \end{enumerate}
  \end{mdframed}
  \subsection{Systemantwort}
  Die Sprung-/Impulsantwort beschreibt Systemantwort vollständig
  \begin{mdframed}[style=exercise]
    \begin{align}
      y(t) &= \int^{\infty}_{-\infty} a(t-\tau) x'(\tau) \dd{\tau}\\
      a(t-\tau) &= W\qty{s(t-\tau)}\nonumber
    \end{align}
  \end{mdframed}
  \begin{mdframed}[style=exercise]
    \begin{align}
      y(t) &= \int_{-\infty}^{\infty} h(t-\tau) x(\tau) \dd{\tau}\\
      h(t-\tau) &= W\qty{\delta(t-\tau)}\nonumber
    \end{align}
  \end{mdframed}
  \subsection{Abtasttheorem}
  Durch die Abtastung wird das Spektrum von $f(t)$ unendlich oft um die Frequenzen $n\cdot \omega_a$ reproduziert.
  \begin{mdframed}[style=exercise]
    \begin{align}
      F_A(\omega) &= \frac{1}{T_A} \sum_{n=-\infty}^{\infty} F(\omega-n\omega_A)\\
      2\omega_g &\leq \omega_A\nonumber
    \end{align}
  \end{mdframed}
  \section{Transformationen}
  \subsection{Fourierreihe}
  \begin{mdframed}[style=exercise]
    \begin{align}
      f(t) &= \sum_{n=0}^{\infty} \qty[a_n \cos(n \omega_0 t) + b_n \sin(n \omega_0 t)]\\
      a_n &= \frac{2}{T}\int_{-T/2}^{T/2}f(t)\cos(n\omega_0 t)\dd{t}\nonumber\\
      b_n &= \frac{2}{T}\int_{-T/2}^{T/2}f(t)\sin(n\omega_0 t)\dd{t}\nonumber
    \end{align}
  \end{mdframed}
  \pagebreak
  \subsection{Fourierreihe, komplex}
  \begin{mdframed}[style=exercise]
    \begin{align}
      f(t) &= \sum_{n=-\infty}^{\infty} c_n e^{jn\omega_0 t}\\
      c_n &= \frac{1}{T} \int_{-T/2}^{T/2} f(t) e^{-jn\omega_0 t}\dd{t}\nonumber
    \end{align}
  \end{mdframed}
  \subsection{Fourierintegral}
  \begin{mdframed}[style=exercise]
    \begin{align}
      f(t) &= \frac{1}{2\pi} \int_{-\infty}^{\infty} F(\omega) e^{j\omega t} \dd{\omega}\\
      F(\omega) &= \int_{-\infty}^{\infty} f(t) e^{-j\omega t} \dd{t}
    \end{align}
  \end{mdframed}
  \subsubsection{Eigenschaften}
  \begin{mdframed}[style=exercise]
    \begin{enumerate}
      \item Linearität\\
      $a f_1(t) + b f_2(t) \laplace a F_1(\omega) + b F_2(\omega)$
      \item Zeitverschiebung\\
      $f(t-t_0) \laplace F(\omega)e^{-j\omega t_0}$
      \item Frequenzverschiebung\\
      $f(t) e^{\pm j\omega_0 t}\laplace F(\omega\mp \omega_0)$
      \item Faltung\\
      $f_1(t)*f_2(t)\laplace F_1(\omega)\cdot F_2(\omega)$\\
      $f_1(\omega)\cdot f_2(\omega)\laplace \frac{1}{2\pi}F_1(t)*F_2(t)$
    \end{enumerate}
  \end{mdframed}
  \subsection{DFT}
  \begin{mdframed}[style=exercise]
    \begin{align}
      x_n &= \frac{1}{N} \sum_{k=0}^{N-1} X_k\cdot e^{i 2 \pi k n / N}\\
      X_k &= \sum_{n=0}^{N-1} x_n\cdot e^{-i 2 \pi k n / N}
    \end{align}
  \end{mdframed}
  \pagebreak
  \subsection{Hilbert Transformation}
  \begin{mdframed}[style=exercise]
    \begin{align}
      x_{\mathrm{ht}}(t) &= x_{\mathrm{r}}(t) * h(t)\\
      H(\omega) &= -j \, \text{sgn}(\omega)
    \end{align}
  \end{mdframed}
  \subsection{z Transformation}
  \begin{mdframed}[style=exercise]
    \begin{align}
      X(z) &= \sum_{n=-\infty}^{\infty} x(n) z^{-n}\\
      x(n) &= \frac{1}{2\pi j} \oint_c X(Z) z^{n-1}\dd{z}
    \end{align}
  \end{mdframed}
  \subsubsection{Übertragungsfunktion}
  \begin{mdframed}[style=exercise]
    \begin{align}
      H(Z) &=\frac{Y(Z)}{X(z)}=\frac{\sum_{k=0}^q b_k z^{-k}}{\sum_{k=0}^p a_k z^{-k}}=k\frac{\prod_{k=1}^q (1-z_k z^{-1})}{\prod_{k=1}^p (1-p_k z^{-1})}
    \end{align}
  \end{mdframed}
  \subsubsection{Verschiebung im Zeitbereich}
  \begin{mdframed}[style=exercise]
    \begin{align}
      Y(z) &= \sum_{n=0}^{\infty} \qty[x(n-m)] z^{-n} =  z^{-m} X(z)\\
      Y(z) &= \sum_{n=0}^{\infty}\qty[x(n+m)]z^{-n} = z^{m}\qty[x(t)-\sum_{n=0}^{m-1} x(n) z^{-n}]
    \end{align}
  \end{mdframed}
  \section{Filter}
  \subsection{FIR}
  \begin{mdframed}[style=exercise]
    \begin{align}
      y[n] &= \sum_{k=0}^{q} b_k x(n-k)
    \end{align}
  \end{mdframed}
  \subsection{IIR}
  \begin{mdframed}[style=exercise]
    \begin{align}
      y[n] &= \sum_{k=0}^{q} b_k x(n-k) - \sum_{k=1}^{p} a_k y(n-k)
    \end{align}
  \end{mdframed}
  \section{Entropie}
  \subsection{Informationsgehalt}
  \begin{mdframed}[style=exercise]
    \begin{align}
      I_i &= \log_2 \frac{1}{p_i}
    \end{align}
  \end{mdframed}
  \subsection{Entropie}
  Mittlerer Informaionsgehalt
  \begin{mdframed}[style=exercise]
    \begin{align}
      H(s) &= E\qty{I} = -\sum_{i=0}^{n-1} p_i \log_2 p_i\\
      0 &\leq H(s) \leq \log_2 n
    \end{align}
  \end{mdframed}
  \begin{mdframed}[style=exercise]
    \begin{align}
      x_1 = 2
    \end{align}
  \end{mdframed}
  Für $H(s)=\log_2 n$ Gleichverteilung und völlige Ungewissheit.


  


\end{document}